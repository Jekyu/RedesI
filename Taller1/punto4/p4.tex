\section{\textbf{Proyecto integrador con IA}}
Una universidad planea implementar una red híbrida:
    \begin{itemize}
        \item Ethernet (IEEE 802.3) para laboratorios de alta capacidad.
        \item WiFi (IEEE 802.11) para áreas comunes.
    \end{itemize}

    \textbf{Requisitos}
    \begin{enumerate}
        \item 5000 usuarios simultáneos.
        \item Soporte a videoconferencia con <100 ms de latencia.
        \item Protección contra ataques ICMP.
        \item Priorización de tráfico académico sobre recreativo.
    \end{enumerate}

\subsection{\textbf{Plan de direccionamiento}}
Se propone un plan de direccionamiento IPv4 privado con subredes, usando la red privada 10.0.0.0/8 para tener un amplio rango de direcciones.\\

Subredes propuestas:

\begin{itemize}
    \item {Red administrativa y servidores}:
        \subitem 10.0.1.0/24
    \item {Laboratorios (Ethernet)}:
        \subitem 10.0.2.0/23 a 10.0.10.0/23 (varias subredes según laboratorios)
    \item {WiFi áreas comunes}:
        \subitem 10.0.20.0/22 (para hasta ~1000 usuarios por segmento)
    \item {WiFi invitados}:
        \subitem 10.0.30.0/24
    \item {Dispositivos IoT y control}:
        \subitem 10.0.40.0/24
    \item {Infraestructura de red (routers, switches, AP)}:
        \subitem 10.0.254.0/24\\
\end{itemize}

Con este esquema se soportan más de 5000 usuarios, con espacio para expansión.

\subsection{\textbf{Políticas de seguridad informática}}

Diseñar políticas de seguridad informática (firewall, segmentación de redes IP, IDS/IPS).\\

%Hay que hacer un prototipo (escenario pequeño con redes IEEE 802.3 e IEEE 802.11), donde se configuren VLMS e implemente (firewall, segmentación de redes IP, IDS/IPS ).
%

\subsubsection{\textbf{Arquitectura de Seguridad General}}
Defensa en Profundidad:

\begin{itemize}
    \item \textbf{Capa 1:} Firewall perimetral
    \item \textbf{Capa 2:} Segmentación con VLANs
    \item \textbf{Capa 3:} IDS/IPS interno
    \item \textbf{Capa 4:} Seguridad en endpoints
\end{itemize}

\subsubsection{\textbf{Configuración Detallada de Firewall}}

\subsubsection*{Bloqueo ICMP específico}
\begin{minted}[fontsize=\small, linenos, bgcolor=lightgray]{python}
    # Ping flood
    deny icmp any any echo-request
    # Respuestas no solicitadas
    deny icmp any any echo-reply
    # ICMP interno permitido
    allow icmp 10.0.0.0/16 any
\end{minted}

\subsubsection*{Acceso administrativo}
\begin{minted}[fontsize=\small, linenos, bgcolor=lightgray]{python}
    # SSH solo desde red admin
    allow tcp 10.0.0.0/24 any eq 22
    # HTTPS admin
    allow tcp 10.0.0.0/24 any eq 443
\end{minted}


\subsubsection*{Servicios públicos}
\begin{minted}[fontsize=\small, linenos, bgcolor=lightgray]{python}
    # Web HTTP/HTTPS
    allow tcp any any eq 80,443
    # DNS
    allow udp any any eq 53
    # NTP
    allow udp any any eq 123
\end{minted}


\subsubsection*{Videoconferencia}
\begin{minted}[fontsize=\small, linenos, bgcolor=lightgray]{python}
    # RTP/RTSP
    allow tcp any any range 5000-6000
    # Puertos multimedia
    allow udp any any range 10000-20000
\end{minted}


\subsubsection*{Bloqueo general}
\begin{minted}[fontsize=\small, linenos, bgcolor=lightgray]{python}
    # Denegar todo lo no permitido
    deny ip any any
\end{minted}

\subsubsection{\textbf{Reglas de Salida (Outbound)}}

\subsubsection*{Tráfico académico prioritario}
\begin{minted}[fontsize=\small, linenos, bgcolor=lightgray]{python}
    allow tcp 10.0.0.0/16 any
    eq 80,443,22,23,25,110,143
    allow udp 10.0.0.0/16 any
    eq 53,123,161,162
\end{minted}

\subsubsection*{Restricciones WiFi invitados}
\begin{minted}[fontsize=\small, linenos, bgcolor=lightgray]{python}
allow tcp 10.0.32.0/21 any eq 80,443,53
deny tcp 10.0.32.0/21 10.0.0.0/16
# No acceso a red interna
\end{minted}



\subsection{\textbf{Mecanismos de calidad de Servicio}}
Implementar mecanismos de calidad de Servicio, Quality Of servive , QoS. (ej. DSCP, colas de prioridad) en el prototipo (escenario de redes IEEE 802.3 e IEEE 802.11).

\subsection{\textbf{Esquema visual de la red}}
Aplicación IA: solicitar a una herramienta de IA la generación de un esquema visual de la red y presente algunas características de gestión de red (velocidades de Tx, desempeños, errores, ataques, logs…entre otros).
