\section{Laboratorio práctico con ping, Wireshark}
\subsection{Configuración LAN (WLAN)}
Se configura la red LAN con IPv4 privadas.

\subsection{Ping}
Se ejecuta pruebas de ping entre ambas bajo IEEE 802.3 (cable) y IEEE 802.11 (WiFi). Los pings se ejecutan de diferentes tamaños (Ej: ping -t x.x.x.x -l 60000).

%Haga una tabla y explique las diferencias que hay entre diferentes velocidades mínimo tres (3), entre las velocidades teóricas (ver tablas No. 1 y 2) y las velocidades obtenidas en la ejecución de los Pings.
% Nota 1: Hay que considerar las tablas No.1 y No. 2, con el fin de evaluar pertinentemente el estándar relacionado con las normas IEEE 802.3 y IEEE 802.11, ya que se presentan cambios con las velocidades y la técnica de modulación.
\subsection{Wireshark}
Se captura el tráfico con Wireshark

%Explique:
% Encabezado IPv4. Explicar de acuerdo al montaje (instalación) realizada en la parte 1 los valores obtenidos.
% Campos de ICMP (Echo Request / Echo Reply). Cuando ejecuto los pings.
% Tiempos de ida y vuelta (RTT). Cuando ejecuto los pings.
% Compare velocidades teóricas vs. Prácticas, cambiando el tamaño del ping

\subsection{Aplicación IA}
Se exportar la captura. pcap y usar una herramienta de IA que explique lospatrones de tráfico, anomalías y latencias.
