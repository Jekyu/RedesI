\section{Metodología}

La metodología adoptada para el desarrollo del sistema de asignación y distribución de medicamentos se estructura en cinco fases fundamentales. Estas fases abarcan desde el análisis inicial del proceso hasta la validación del prototipo final, integrando tecnologías modernas, principios de ingeniería de software y técnicas avanzadas de inteligencia artificial. El enfoque metodológico se orienta principalmente a mejorar la eficiencia operativa, reducir los costos del proceso logístico y brindar una experiencia más ágil y confiable para los pacientes.

\subsection{Fase 1: Análisis del Proceso Actual}

La primera fase consiste en un estudio profundo del funcionamiento actual de la dispensación de medicamentos dentro de un centro médico. Esta etapa incluye la observación directa del flujo de trabajo, la revisión de protocolos existentes y la caracterización de los pasos que sigue un paciente desde la emisión de una receta hasta la entrega del medicamento. Durante este análisis se identifican los puntos críticos como los tiempos prolongados de espera en la farmacia, la congestión en áreas de entrega, las demoras en la preparación de pedidos y la frecuencia de errores humanos derivados de la interpretación de recetas manuscritas. Paralelamente, se realizan entrevistas y encuestas al personal de la farmacia, médicos y pacientes, con el fin de comprender sus principales dificultades y validar necesidades reales. Esta información permite construir un diagnóstico claro que servirá como referencia para proponer soluciones que atiendan tanto a la reducción de costos como a la optimización del servicio.

\subsection{Fase 2: Modelado del Sistema y Definición de Requerimientos}

Una vez identificadas las debilidades del proceso actual, se procede a modelar el sistema objetivo mediante metodologías formales. En esta fase se definen los requerimientos funcionales y no funcionales que guiarán el desarrollo del sistema. Se elaboran diagramas UML, como casos de uso, diagramas de actividad y diagramas de secuencia, que permiten representar de manera precisa la interacción entre los distintos actores del sistema: pacientes, personal de farmacia, mensajeros y módulos automatizados. Además, se diseña el modelo de datos que integrará la información relevante, tales como recetas médicas, estado de pedidos, inventario disponible, rutas de entrega y perfiles de usuarios. Este modelado contribuye a garantizar que el sistema propuesto sea interoperable, escalable y seguro, y que atienda de forma adecuada los flujos identificados durante el análisis inicial.

\subsection{Fase 3: Diseño de la Arquitectura Tecnológica}

En la tercera fase se construye la arquitectura tecnológica que soportará todos los componentes del sistema. Esta arquitectura incluye una plataforma \textit{backend} basada en servicios REST, encargada de gestionar las operaciones esenciales como la validación de recetas, el registro de pedidos, la administración de inventarios y la comunicación entre pacientes y farmacia. Asimismo, se diseña una aplicación móvil destinada a los pacientes, desde la cual pueden cargar la receta digital, realizar el pedido, efectuar el pago y hacer seguimiento al estado del proceso. Por su parte, el personal de la farmacia contará con un panel administrativo en el que podrán visualizar los pedidos entrantes, cambiar su estado y coordinar la entrega. La arquitectura incorpora protocolos seguros de comunicación (como HTTPS y OAuth 2.0), además de mecanismos de cifrado para proteger información sensible. El objetivo principal es disponer de una infraestructura robusta y flexible que permita la integración con módulos de inteligencia artificial y futuras ampliaciones del sistema.

\subsection{Fase 4: Integración de Técnicas de Inteligencia Artificial}

La cuarta fase introduce la inteligencia artificial como un componente clave para la automatización y optimización del sistema. En primer lugar, se incluye un módulo de clasificación automática de recetas, basado en técnicas de visión computacional y modelos de reconocimiento óptico de caracteres (OCR) mejorados con redes neuronales. Este módulo permite extraer información relevante de las recetas, validar su contenido y asociarlo con el inventario disponible, reduciendo los errores que se presentan en la lectura manual. En segundo lugar, se implementan modelos de aprendizaje automático orientados a la predicción de demanda de medicamentos. Estos modelos analizan patrones históricos de consumo, variaciones por temporada y tendencias por grupo poblacional, permitiendo anticipar picos de demanda y optimizar los niveles de inventario, lo cual contribuye a la reducción de costos operativos, tal como lo señalan estudios recientes \cite{suwatcharachaitiwong2020medication, potters2024coldchain}.

Además, se integran algoritmos inteligentes de optimización de rutas, como algoritmos genéticos, colonia de hormigas y técnicas de aprendizaje por refuerzo profundo. Estos métodos permiten generar rutas de entrega más eficientes, reducir tiempos de desplazamiento y minimizar el uso de recursos logísticos, siguiendo enfoques igualmente observados en la literatura \cite{lucchese2019healthlogistics, magalhaes2006pharma}. De manera complementaria, se introduce un sistema de recomendación que, en función del historial del paciente, genera recordatorios automáticos, sugiere fechas óptimas de reposición e incluso plantea planes periódicos de entrega, facilitando adherencia al tratamiento y mejorando la experiencia general del usuario.

\subsection{Fase 5: Desarrollo del Prototipo y Validación}

La fase final consiste en la implementación del prototipo funcional del sistema y su validación mediante pruebas controladas. El desarrollo se basa en las especificaciones definidas previamente, asegurando la correcta integración entre la aplicación móvil, el panel administrativo y los servicios del \textit{backend}. Se realizan pruebas funcionales y de rendimiento para verificar que los módulos de IA, el manejo de inventario, los procesos de registro de pedidos y los flujos de comunicación operen de manera consistente. Posteriormente, se llevan a cabo pruebas piloto con usuarios reales o simulados, evaluando indicadores como la reducción del tiempo total de dispensación, la disminución de errores humanos, la eficiencia en la asignación de rutas de entrega y el nivel de satisfacción de los pacientes. Finalmente, los resultados obtenidos se comparan con los datos del proceso tradicional, permitiendo medir de forma objetiva el impacto del sistema desarrollado y validar su potencial para implementarse en un entorno real.
