\section{Formulación del Problema}

En la actualidad, los usuarios de centros médicos, clínicas y hospitales (especialmente pacientes crónicos, de la tercera edad o con movilidad reducida) se enfrentan a una serie de inconvenientes al momento de adquirir sus medicamentos recetados. Estos problemas incluyen:

\begin{itemize}
    \item \textbf{Tiempos de Espera Prolongados:} Después de una consulta, el paciente debe dirigirse a la farmacia, hacer una fila para entregar la receta, esperar a que preparen el pedido y luego hacer otra fila para pagar. Esto puede agregar más de una hora a su visita.
    \item \textbf{Congestión en Áreas Comunes:} Las farmacias de las centrales médicas suelen estar abarrotadas, generando estrés tanto para los pacientes como para el personal.
    \item \textbf{Dificultad para Pacientes con Movilidad Reducida:} Para personas mayores, discapacitadas o post-operadas, el desplazamiento físico dentro de un gran complejo médico puede ser una tarea difícil y dolorosa.
    \item \textbf{Ineficiencia en la Gestión de Pedidos:} El sistema tradicional de entrega manual de recetas y preparación bajo demanda es propenso a errores humanos (lectura de recetas, equivocación de medicamento) y no está optimizado para la gestión de múltiples pedidos de manera simultánea.
    \item \textbf{Falta de Seguimiento:} El paciente no tiene visibilidad del estado de su pedido una vez entregada la receta, generando incertidumbre.
\end{itemize}

Estos problemas evidencian la necesidad de un sistema que optimice el proceso de dispensación y entrega de medicamentos, mejorando la experiencia del usuario y la eficiencia operativa de la central médica mediante el uso de la tecnología.

\section{Objetivos}
\label{sec:objetivos}

\subsection{Objetivo General}
Diseñar e implementar un prototipo de aplicación móvil y un sistema \textit{backend} que optimice el proceso de solicitud y entrega de medicamentos en una central médica.

\subsection{Objetivos Específicos}
\label{subsec:objetivos-especificos}

\begin{enumerate}
    \item Desarrollar un módulo de usuario final (aplicación móvil) que permita a los pacientes:
    \begin{itemize}
        \item Escanear o cargar digitalmente su receta médica.
        \item Realizar el pedido de sus medicamentos de forma remota.
        \item Realizar el pago de forma electrónica integrada.
        \item Recibir notificaciones en tiempo real sobre el estado de su pedido.
        \item Solicitar la entrega del pedido en un punto específico.
    \end{itemize}
    
    \item Desarrollar un módulo de administración (aplicación web o de escritorio) para el personal de la farmacia que permita:
    \begin{itemize}
        \item Visualizar un \textit{dashboard} con los pedidos entrantes en tiempo real.
        \item Gestionar el estado de cada pedido.
        \item Notificar automáticamente al usuario.
        \item Gestionar una flota de mensajeros/entregadores.
    \end{itemize}
    
    \item Diseñar y modelar la arquitectura de red y comunicación del sistema, definiendo:
    \begin{itemize}
        \item Los protocolos de comunicación.
        \item La estructura de la base de datos.
        \item Los mecanismos de seguridad para proteger los datos sensibles.
    \end{itemize}
  
\end{enumerate}
