\small \textbf{
Resumen--La asignación y distribución eficiente de medicamentos constituye un proceso fundamental para el adecuado funcionamiento de los sistemas de salud, pues influye en la continuidad del tratamiento, la experiencia del paciente y los costos operativos de las instituciones médicas. Los métodos tradicionales de dispensación presentan problemas como tiempos de espera prolongados, congestión en farmacias, errores humanos y falta de visibilidad sobre el estado del pedido. En respuesta, investigaciones recientes incluyendo modelos de optimización logística, sistemas IoT, mecanismos automáticos de dispensación y esquemas híbridos de entrega han demostrado reducciones significativas en costos y mejoras en la eficiencia del servicio. Este proyecto propone el diseño de una aplicación móvil y un sistema \textit{backend} que optimizan la solicitud, preparación y entrega de medicamentos, con un enfoque en la disminución de costos y la mejora de la accesibilidad para pacientes con movilidad reducida. La propuesta se fundamenta en estudios de revistas indexadas Q1 y Q2, validando la pertinencia y solidez académica del desarrollo planteado.}\\

\begin{abstract}
The efficient allocation and distribution of medications is a critical component of modern healthcare systems, directly influencing treatment continuity, patient satisfaction, and operational costs. Traditional medication dispensing processes in hospitals and medical centers often suffer from long waiting times, bottlenecks in pharmacy areas, human errors during prescription handling, and limited transparency for patients. Recent advances in health logistics, including automated dispensing systems, Internet of Things (IoT) solutions, hybrid delivery models, and route optimization techniques, have shown significant potential to reduce operational costs and improve service quality. This project proposes the design and development of a mobile application and backend platform aimed at optimizing the medication request, preparation, and delivery workflow. Supported by studies published in high-impact journals (Q1–Q2), this work emphasizes cost reduction, process automation, and improved accessibility for patients with mobility limitations. The resulting system aims to streamline operations within medical centers and provide a more efficient and user-centered pharmaceutical service.
\end{abstract}

\section{Introducción}

La correcta asignación y distribución de medicamentos es un componente esencial para el funcionamiento eficiente de los sistemas de salud, ya que impacta directamente en la continuidad del tratamiento, la experiencia del paciente y los costos operativos de clínicas y hospitales. Los métodos tradicionales de dispensación presentan múltiples limitaciones como tiempos de espera prolongados, congestión en farmacias, errores humanos en la preparación de pedidos y escasa visibilidad del estado de la solicitud.

La literatura reciente propone diversas soluciones tecnológicas y logísticas para abordar estos desafíos. Por ejemplo, los modelos híbridos de distribución que combinan conveniencia, lockers y entrega a domicilio han demostrado mejorar la eficiencia operativa y reducir costos en procesos de última milla \cite{suwatcharachaitiwong2020medication}. A nivel de cadena de suministro, la integración de sincronización de medicamentos con esquemas mixtos de entrega permite reducir costos en cadenas farmacéuticas sensibles, especialmente aquellas que requieren control de temperatura \cite{potters2024coldchain}.

Asimismo, los estudios sobre optimización logística en servicios de salud muestran que técnicas como el \textit{location-routing} y la programación matemática contribuyen a minimizar costos y mejorar la disponibilidad de medicamentos \cite{lucchese2019healthlogistics}. En contextos operativos reales, la incorporación de modelos de ruteo dinámico ha demostrado reducir tiempos de entrega y mejorar la capacidad de respuesta en la distribución farmacéutica \cite{magalhaes2006pharma}. De manera complementaria, soluciones centradas en la entrega urgente de medicamentos y estrategias de última milla fortalecen la atención de pacientes con limitaciones de movilidad o necesidades críticas \cite{kritchanchai2024delivery}.

Estos estudios evidencian que la incorporación de tecnologías avanzadas, modelos de optimización y sistemas de información permite diseñar mecanismos de distribución de medicamentos más eficientes, económicos y centrados en el paciente. En este proyecto se propone el diseño de un prototipo de aplicación móvil y un sistema \textit{backend} que optimicen la solicitud, preparación y entrega de medicamentos, fundamentado en la literatura científica de alto impacto (Q1–Q2) y orientado principalmente a la reducción de costos y la mejora de la experiencia del usuario.


\subsection{Formulación del Problema}

En la actualidad, los usuarios de centros médicos, clínicas y hospitales (especialmente pacientes crónicos, de la tercera edad o con movilidad reducida) se enfrentan a una serie de inconvenientes al momento de adquirir sus medicamentos recetados. Estos problemas incluyen:

\begin{itemize}
    \item \textbf{Tiempos de Espera Prolongados:} Después de una consulta, el paciente debe dirigirse a la farmacia, hacer una fila para entregar la receta, esperar a que preparen el pedido y luego hacer otra fila para pagar. Esto puede agregar más de una hora a su visita.
    \item \textbf{Congestión en Áreas Comunes:} Las farmacias de las centrales médicas suelen estar abarrotadas, generando estrés tanto para los pacientes como para el personal.
    \item \textbf{Dificultad para Pacientes con Movilidad Reducida:} Para personas mayores, discapacitadas o post-operadas, el desplazamiento físico dentro de un gran complejo médico puede ser una tarea difícil y dolorosa.
    \item \textbf{Ineficiencia en la Gestión de Pedidos:} El sistema tradicional de entrega manual de recetas y preparación bajo demanda es propenso a errores humanos (lectura de recetas, equivocación de medicamento) y no está optimizado para la gestión de múltiples pedidos de manera simultánea.
    \item \textbf{Falta de Seguimiento:} El paciente no tiene visibilidad del estado de su pedido una vez entregada la receta, generando incertidumbre.
\end{itemize}

Estos problemas evidencian la necesidad de un sistema que optimice el proceso de dispensación y entrega de medicamentos, mejorando la experiencia del usuario y la eficiencia operativa de la central médica mediante el uso de la tecnología.

\subsection{Objetivos}
\label{sec:objetivos}

\subsubsection{Objetivo General}
Diseñar e implementar un prototipo de aplicación móvil y un sistema \textit{backend} que optimice el proceso de solicitud y entrega de medicamentos en una central médica.

\subsubsection{Objetivos Específicos}
\label{subsec:objetivos-especificos}

\begin{enumerate}
    \item Desarrollar un módulo de usuario final (aplicación móvil) que permita a los pacientes:
    \begin{itemize}
        \item Escanear o cargar digitalmente su receta médica.
        \item Realizar el pedido de sus medicamentos de forma remota.
        \item Realizar el pago de forma electrónica integrada.
        \item Recibir notificaciones en tiempo real sobre el estado de su pedido.
        \item Solicitar la entrega del pedido en un punto específico.
    \end{itemize}

    \item Desarrollar un módulo de administración (aplicación web o de escritorio) para el personal de la farmacia que permita:
    \begin{itemize}
        \item Visualizar un \textit{dashboard} con los pedidos entrantes en tiempo real.
        \item Gestionar el estado de cada pedido.
        \item Notificar automáticamente al usuario.
        \item Gestionar una flota de mensajeros/entregadores.
    \end{itemize}

    \item Diseñar y modelar la arquitectura de red y comunicación del sistema, definiendo:
    \begin{itemize}
        \item Los protocolos de comunicación.
        \item La estructura de la base de datos.
        \item Los mecanismos de seguridad para proteger los datos sensibles.
    \end{itemize}

\end{enumerate}
