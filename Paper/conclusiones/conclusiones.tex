\section{Conclusiones}

El desarrollo del aplicativo web basado en microservicios permitió comprobar la efectividad de una arquitectura modular y distribuida en escenarios donde se requieren sistemas robustos, mantenibles y escalables. La metodología empleada, acompañada de prácticas de diseño orientadas a componentes desacoplados, aseguró que cada servicio pudiera evolucionar de manera independiente sin comprometer la estabilidad del conjunto.\\

La incorporación de inteligencia artificial representa un aporte significativo al valor del sistema, ya que abre la puerta a mecanismos automatizados que optimizan el flujo de trabajo, reducen la intervención manual y generan análisis predictivos útiles para la toma de decisiones. Este enfoque permitirá que futuras versiones del sistema incluyan algoritmos más complejos para detección de anomalías, análisis de tendencias, clasificación de eventos y automatización de procesos de negocio.\\

Adicionalmente, se concluye que la estructura de proyecto implementada facilita la estandarización del ciclo de vida del software, desde la fase de diseño hasta el despliegue en ambientes productivos. La integración de servicios, documentación centralizada, pruebas distribuidas y componentes de infraestructura ayudan a mantener una línea clara de desarrollo y escalar el sistema conforme crezcan las necesidades.\\

Finalmente, el enfoque adoptado demuestra que la combinación de microservicios, buenas prácticas de desarrollo y capacidades de IA permite la construcción de aplicaciones modernas preparadas para entornos altamente dinámicos y exigentes. Se recomienda continuar esta línea de trabajo profundizando en aspectos como tolerancia a fallos, observabilidad avanzada, seguridad basada en políticas y expansión del módulo de inteligencia artificial.\\
