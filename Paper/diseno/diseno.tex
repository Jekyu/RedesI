\section{Diseño}

El diseño del sistema parte de la premisa de ofrecer una plataforma web escalable y segura que permita a pacientes, personal de farmacia y gestores de logística coordinar la solicitud, preparación y entrega de medicamentos. La arquitectura propuesta es de tipo \textit{microservicios} con las siguientes capas principales: interfaz de usuario (frontend web), API gateway / backend (servicios REST), capa de procesamiento de IA, bases de datos y subsistema de mensajería/colas para tareas asíncronas y notificaciones.\\

En el frontend se propone una aplicación web responsiva desarrollada con React (o Vue) que proporcione flujos claros para:
\begin{itemize}
    \item Captura o carga de receta.
    \item Visualización de inventario y precios.
    \item Realización de pedidos y pago.
    \item Seguimiento en tiempo real del estado del pedido.
\end{itemize}
 El diseño de la UI seguirá principios de accesibilidad (WCAG) para asegurar usabilidad por parte de pacientes con movilidad reducida o limitaciones sensoriales.\\

El backend se organiza en microservicios para facilitar escalabilidad y despliegue independiente:
\begin{itemize}
    \item \textbf{auth-service}: gestión de usuarios, roles y tokens
    \item \textbf{orders-service}: pedido y estado
    \item \textbf{inventory-service}: stock y caducidades
    \item \textbf{routing-service}: generación de rutas y asignación de mensajeros
    \item \textbf{notification-service}: notificaciones push, correo y SMS
    \item \textbf{ai-service}: OCR, predicción de demanda y optimización de rutas.
\end{itemize}
 Los servicios exponen APIs REST (JSON) y se comunican internamente por HTTP/HTTPS y, cuando se requiere baja latencia o mensajes asíncronos, por una cola (RabbitMQ / Kafka).\\

En la capa de datos se propone usar PostgreSQL como base de datos relacional principal por su robustez y ACID; Redis para cache (sesiones, información de carrito) y como broker para tareas rápidas; y un almacén de objetos (S3 compatible) para recetas escaneadas y archivos. Para los modelos de IA y trabajos batch (reentrenamiento, scoring masivo) se emplea un cluster ML separado (puede ser un entorno Kubernetes con nodos GPU o instancias en la nube).

Desde el punto de vista de la seguridad y cumplimiento, se recomienda cifrado TLS en tránsito, cifrado en reposo para datos sensibles, control de acceso basado en roles (RBAC), registro de auditoría y la implementación de prácticas de privacidad (minimización de datos, consentimiento explícito) para cumplir marcos regulatorios locales aplicables (por ejemplo consideraciones tipo HIPAA cuando corresponda).

Finalmente, el sistema incluye paneles de observabilidad (Prometheus + Grafana), trazas distribuidas (OpenTelemetry) y alertas (PagerDuty/opsgenie) para operar la plataforma con garantías de disponibilidad y respuesta ante incidentes.
